%!TEX TS-program = xelatex
%!TEX encoding = UTF-8 Unicode
\PassOptionsToPackage{AutoFakeBold=true,AutoFakeSlant=true}{xeCJK}

\documentclass[12pt,openany]{book}
\usepackage[a4paper]{geometry}
\usepackage{graphicx}
\usepackage{amssymb,amsmath,amsthm}	
\usepackage{booktabs}				
\usepackage{hyperref}				
\usepackage{titlesec, titletoc}

\usepackage{xeCJK}



% 设置主要字体:正文 (mainfont)、无衬线字体 (sansfont) 和等宽字体 (monofont)。
\setCJKmainfont[Path=/Volumes/THAWSPACE/CshProject/code_matrix.git/code/LaTex/fontFiles/方正/]{FangZhengShuSong-GBK-1.ttf}
\setCJKsansfont[Path=/Volumes/THAWSPACE/CshProject/code_matrix.git/code/LaTex/fontFiles/方正/]{FangZhengHeiTi-GBK-1.ttf}
\setCJKmonofont[Path=/Volumes/THAWSPACE/CshProject/code_matrix.git/code/LaTex/fontFiles/方正/]{FangZhengFangSong-GBK-1.ttf}

%xeCJK(推荐的新方式)
\newCJKfontfamily\kai[Path=/Volumes/THAWSPACE/CshProject/code_matrix.git/code/LaTex/fontFiles/方正/]{FangZhengKaiTi-GBK-1.ttf}
\newCJKfontfamily\fsong[Path=/Volumes/THAWSPACE/CshProject/code_matrix.git/code/LaTex/fontFiles/方正/]{FangZhengFangSong-GBK-1.ttf}
\newCJKfontfamily\hei[Path=/Volumes/THAWSPACE/CshProject/code_matrix.git/code/LaTex/fontFiles/方正/]{FangZhengHeiTi-GBK-1.ttf}
\newCJKfontfamily\ssong[Path=/Volumes/THAWSPACE/CshProject/code_matrix.git/code/LaTex/fontFiles/方正/]{FangZhengShuSong-GBK-1.ttf}
\newCJKfontfamily\FZJiaGuWen[Path=/Volumes/THAWSPACE/CshProject/code_matrix.git/code/LaTex/fontFiles/方正/]{方正甲骨文.ttf}

\setromanfont[Path=/Volumes/THAWSPACE/Soft.Ok/texlive/2024/texmf-dist/fonts/opentype/public/tex-gyre/
, Extension=.otf, UprightFont=*-Regular,BoldFont=*-Bold, ItalicFont=*-Italic, BoldItalicFont=*-BoldItalic
,Mapping=tex-text]{texgyrepagella}%TeX Gyre Pagella

\setsansfont[Path=/Volumes/THAWSPACE/Soft.Ok/texlive/2024/texmf-dist/fonts/opentype/public/tex-gyre/
, Extension=.otf, UprightFont=*-Regular,BoldFont=*-Bold, ItalicFont=*-Italic, BoldItalicFont=*-BoldItalic
              ,Scale=MatchLowercase,Mapping=tex-text]{texgyrepagella}

\setlength{\parindent}{0pt}  	
\linespread{1.2} 				
\setlength{\parskip}{15pt} 		

\renewcommand\contentsname{目~录~}
\renewcommand\listfigurename{图~列~表~}
\renewcommand\listtablename{表~目~录~}

%=========================制作拇指索引===================
\usepackage{tikz,pgf}
\usetikzlibrary{shapes,calc}
\makeatletter
\newcommand\thumb{%
  \if@mainmatter
  \begingroup
  \catcode`\$=3
  \tikzpicture[remember picture,overlay] % thumb index
    \ifodd\value{page}
      \node[fill=gray,text=black,anchor=north east,xshift=2mm,
            yshift=-22mm-\arabic{chapter}*20mm,
            shape=semicircle,shape border rotate=90,
            minimum height=10mm,minimum width=5mm,
            font=\normalfont\sffamily\bfseries\Huge]
        at (current page.north east)
        {\llap{\arabic{chapter}\hspace{1mm}}};
    \else
      \node[fill=gray,text=black,anchor=north west,xshift=-2mm,
            yshift=-22mm-\arabic{chapter}*20mm,
            shape=semicircle,shape border rotate=270,
            minimum height=10mm,minimum width=5mm,
            font=\normalfont\sffamily\bfseries\Huge]
        at (current page.north west)
        {\rlap{\hspace{1mm}\arabic{chapter}}};
    \fi
  \endtikzpicture
  \endgroup
  \fi}
\makeatother

\usepackage{fancyhdr}
\fancypagestyle{plain}{%
\fancyhf{} % clear all header and footer fields
\fancyhead[r]{\thumb} % except the center
\renewcommand{\headrulewidth}{0pt}
\renewcommand{\footrulewidth}{0pt}}
%=========================制作拇指索引===================


\titlecontents{chapter}
[0em]
{}
{\large\CJKfamily{fzsong}{第 \thecontentslabel 章\quad}}
{}{\hfill\contentspage}
%
\titlecontents{section}
[4em]
{}
{\thecontentslabel\quad}
{}{\titlerule*{.} \contentspage}


\titleformat{\chapter}[display]
	{\bfseries\Large}
	{\titlerule[1pt]%
     \filleft%
    \parbox{1cm}{\vbox to 2.5cm{%
    \vfill\hbox to 1cm{\hfill\Huge 第 \hfill}%
    \vfill\hbox to 1cm{\hfill\Huge \thechapter\hfill}%
    \vfill\hbox to 1cm{\hfill\Huge 章 \hfill}%
    }}}
	{1ex}
	{\Huge
	 \filright}
	[{\titlerule[2pt]}]

\title{\bf{podman使用笔记}}
\author{virhuiai}
\date{}

% \usepackage{minted}
\usepackage[newfloat]{minted}%(推荐方式) 除了加载 minted 核心功能外,还通过 newfloat 选项启用 newfloat 包的集成。
% newfloat 包是 caption 包的配套包,专门用于定义新的浮动环境。
% 启用后:
%   listing 环境由 newfloat 包管理,支持更完善的浮动体控制。
%   与 caption 包深度兼容(自动处理标题格式、编号、参考等)。
%   支持 \listoflistings(代码列表)命令,生成所有代码块的目录。
%   即使不加载 float 包,也能正常工作,且更现代化。
\usepackage{caption}
% 可选:美化标题
\captionsetup[listing]{
  name=代码,
  labelsep=quad,
  labelfont=bf
}
\setminted[shell]{
  % linenos=true,
  numbersep=8pt,
  frame=lines,
  framesep=2mm,
  baselinestretch=1.2,
  fontsize=\small,
  bgcolor=gray!5,
  breaklines=true,
  escapeinside=||
}

\begin{document}
\maketitle

\tableofcontents

\chapter{macOS 上Podman的安装配置}



\section{前言}
本文档记录了在 macOS 系统上使用 Homebrew 安装并配置 \textbf{Podman} 的完整流程,包括:
\begin{itemize}
  \item 配置文件持久化(符号链接)
  \item Podman 虚拟机初始化与启动
  \item 镜像源加速配置
  \item 存储路径自定义
  \item 常见问题与优化建议
\end{itemize}

适用于 macOS 15.6,Podman 版本 5.6.2。

\section{环境准备:配置文件持久化}

将用户配置文件迁移至外部存储(如移动硬盘),防止系统重装丢失。

% \begin{listing}[htbp]
\begin{minted}[]{shell}
# 移动 .zshrc 到外部目录
mv ~/.zshrc /Volumes/THAWSPACE/Soft.Ok/devEnv/

# 创建符号链接
ln -s /Volumes/THAWSPACE/Soft.Ok/devEnv/.zshrc ~/.zshrc

# 创建容器配置目录
mkdir -p /Volumes/THAWSPACE/Soft.Ok/devEnv/containers_config
mkdir -p /Volumes/THAWSPACE/Soft.Ok/devEnv/containers_share

# 链接容器配置目录
ln -s /Volumes/THAWSPACE/Soft.Ok/devEnv/containers_config ~/.config/containers
ln -s /Volumes/THAWSPACE/Soft.Ok/devEnv/containers_share ~/.local/share/containers
\end{minted}
% \caption{创建符号链接和容器配置目录}
% \label{code:symlink}
% \end{listing}


\section{Podman 安装与初始化}

\subsection{安装 Podman}
% \begin{lstlisting}[caption={使用 Homebrew 安装 Podman}]
使用 Homebrew 安装 Podman
\begin{minted}[]{shell}
brew install podman
\end{minted}
% \end{lstlisting}

安装输出示例(Podman 5.6.2):
\begin{minted}[]{shell}
==> Fetching podman
==> Downloading https://ghcr.io/v2/homebrew/core/podman/...
==> Pouring podman--5.6.2.sonoma.bottle.tar.gz
==> Caveats
In order to run containers locally, podman depends on a Linux kernel.
One can be started manually using `podman machine` from this package.
To start a podman VM automatically at login, also install the cask "podman-desktop".
==> Summary
🍺 /usr/local/Cellar/podman/5.6.2: 217 files, 79.8MB
\end{minted}

\subsection{初始化并启动虚拟机}

% 推荐使用 \lstinline|podman machine init --now| 一键完成初始化与启动:
% \begin{lstlisting}[caption={初始化并启动 Podman 虚拟机}]
\begin{minted}[]{shell}
podman machine init --now
\end{minted}

等价于:

\begin{minted}[]{shell}
podman machine init
podman machine start
\end{minted}


启动成功后输出:
\begin{minted}[]{shell}
Machine "podman-machine-default" started successfully
API forwarding listening on: /var/run/docker.sock
Docker API clients default to this address. You do not need to set DOCKER_HOST.
\end{minted}

\section{测试运行容器}

% \begin{lstlisting}[caption={运行 hello-world 测试容器}]
\begin{minted}[]{shell}
# 运行 hello-world 测试容器
podman run --rm hello-world
\end{minted}

输出:
\begin{minted}[]{shell}
!... Hello Podman World ...!
         .--"--.
       / - - \
      / (O) (O) \
   ~~~| -=(,Y,)=- |
    .---. /` \ |~~
 ~/ o o \~~~~.----. ~~
  | =(X)= |~ / (O (O) \
   ~~~~~~~ ~| =(Y_)=- |
  ~~~~ ~~~| U |~~
Project: https://github.com/containers/podman
Website: https://podman.io
\end{minted}

\section{镜像源加速}

\begin{minted}[]{shell}
podman pull docker.1ms.run/erlang:28.1.0.0-alpine
\end{minted}

% \subsection{方法一:在虚拟机内修改(推荐)}

% 进入 Podman 虚拟机:
% \begin{lstlisting}[caption={进入 Podman 虚拟机}]
% podman machine ssh
% \end{lstlisting}

% 编辑镜像源配置文件:
% \begin{lstlisting}
% sudo vi /etc/containers/registries.conf.d/999-podman-machine.conf
% \end{lstlisting}

% 内容如下:
% \begin{lstlisting}[language=toml, caption={999-podman-machine.conf}]
% unqualified-search-registries = ["docker.io"]

% [[registry]]
% prefix = "docker.io"
% location = "docker.1ms.run"

% [[registry]]
% prefix = "docker.io"
% location = "hub.rat.dev"

% [[registry]]
% prefix = "docker.io"
% location = "docker.xuanyuan.me"

% [[registry]]
% prefix = "docker.io"
% location = "docker.1panel.live"
% \end{lstlisting}

% \subsection{方法二:宿主机配置(rootless 模式)}

% 在 \lstinline|~/.config/containers/registries.conf| 中添加:

% \begin{lstlisting}[language=toml]
% unqualified-search-registries = ["docker.io"]

% [[registry]]
% prefix = "docker.io"
% location = "docker.mirrors.ustc.edu.cn"
% \end{lstlisting}

% \section{自定义镜像存储路径}

% 编辑 \lstinline|~/.config/containers/storage.conf|:

% \begin{lstlisting}[language=toml, caption={storage.conf 关键配置}]
% [storage]
% graphroot = "/Volumes/THAWSPACE/Soft.Ok/devEnv/podman_containers"
% runroot = "/run/containers/storage"

% [storage.options]
% # 可选:调整 overlay 驱动参数
% \end{lstlisting}

% > \textbf{注意}:\lstinline|graphroot| 为容器镜像和卷的存储根目录。

% \section{进阶:Erlang 镜像拉取测试}

% \begin{lstlisting}[caption={拉取 Erlang 镜像}]
% podman pull docker.1ms.run/erlang:28.1.0.0-alpine
% \end{lstlisting}

% \section{常见问题与建议}

% \begin{itemize}
%   \item \textbf{权限问题}:若需绑定低端口(<1024),执行:
%     \begin{lstlisting}
%     podman machine set --rootful
%     \end{lstlisting}
%   \item \textbf{Docker 兼容性}:设置环境变量:
%     \begin{lstlisting}
%     export DOCKER_HOST='unix:///var/folders/.../podman-machine-default-api.sock'
%     \end{lstlisting}
%     或安装 \lstinline|podman-mac-helper|。
%   \item \textbf{自动启动}:安装 \lstinline|podman-desktop| Cask 实现开机自启。
% \end{itemize}

% \section{参考资料}

% \begin{itemize}
%   \item 官方文档: \url{https://docs.podman.io}
%   \item GitHub: \url{https://github.com/containers/podman}
%   \item Podman Desktop: \url{https://podman-desktop.io}
%   \item 镜像加速: \url{https://blog.csdn.net/adorechen/article/details/147652534}
%   \item 存储配置: \url{https://github.com/containers/podman/blob/main/docs/tutorials/rootless_tutorial.md}
% \end{itemize}




\end{document} 

\begin{minted}{shell}

\end{minted}

\end{document} 